\section*{Research Interests}

High-energy astrophysics, computational physics, data-driven science,
and virtual reality for scientific data visualization and education

%------------------------------------------------------------------------------
\section*{Employment}

\begin{tlist}

\item[2013\,--\,] Research Associate, Steward Observatory

\item[2010\,--\,12] NORDITA Fellow, Nordic Institute for Theoretical Physics

\item[2010] Teaching Fellow, Harvard University

\item[2007\,--\,10] ITC Fellow, Harvard-Smithsonian Center for Astrophysics

\item[2005\,--\,07] Summer Internship in Theory Division,
  Los Alamos National Laboratory

\item[2003\,--\,07] Research Assistant in Astrophysics, University of Arizona

\item[2002\,--\,03, 06] Teaching Assistant in Physics and
  Computational Physics, University of Arizona

\item[2001] Teaching and Research Assistant in Mathematics,
  University of Arizona

\item[2000] Software Developer in Computational Physics,
  Texas A\&M University

\end{tlist}

%------------------------------------------------------------------------------
\section*{Education}

\begin{tlist}

\item[2007] Ph.D. in Physics, University of Arizona

\item[2002] B.S. in Physics and Mathematics (Cum Laude), University of Arizona

\end{tlist}

%------------------------------------------------------------------------------
\section*{Professional and Academic Service}

\begin{tlist}

\item[2015\,--\,] Founder and contributor of the
  \texttt{mockservation} project

\item[2015\,--\,] Co-founder of the AstroCardboard project and
  developer of the \texttt{RosettaTour} app

\item[2009\,--\,] Peer reviewer for ApJ, MNRAS, PASJ, AA, and SIGGRAPH

\item[2012] Co-organizer of the ``Astrophysics Code Comparison
  Workshop'' at NORDITA

\item[2010] Guest lecturer in numerical analysis class on topics of
  GPU computation

\item[2009] Co-organizer of the ``Plasma Astrophysics Meetings'' at ITC

\item[2008\,--\,09] Member of the CfA Postdoc Council

\item[2008] Member of the local organizing committee of the
  ``Saturation and Transport Properties of MRI-driven Turbulence''
  conference at IAS

\item[2007\,--\,08] Organizer of the ``Astrophysical Turbulence
  Meetings'' at ITC

\end{tlist}

%------------------------------------------------------------------------------
\section*{Grants and Awards}

\begin{tlist}

\item[2016\,--\,2017] ``X-ray Variability of Sgr~A* as a Probe of
  Plasma Physics in Accretion Flows'',
  Co-I (PI: Feryal \"Ozel),
  Chandra X-ray Observatory Cycle 17 (Theory)

\item[2013\,--\,2016] ``Multi-Scale Plasma Flows Around Black Holes'',
  named collaborator (PI: Jonathan McKinney),
  NASA/NSF Theoretical and Computational Astrophysics Network

\item[2010\,--\,2012] NORDITA Fellowship

\item[2007\,--\,2010] Harvard ITC Fellowship

\item[2007\,--\,2008] ``Understanding the Flares of Sgr~A* trough 3D
  Radiative Magnetohydrodynamic Simulations'',
  Co-I (PI: Dimitrios Psaltis),
  Chandra X-ray Observatory Cycle 9 (Theory)

\end{tlist}

%------------------------------------------------------------------------------
\section*{Selected Software Projects}

\begin{Tlist}

\item[\texttt{lux}] A versatile, scalable, extendable framework to
  simulate astrophysical systems, fully open sourced once completed

\item[\texttt{mockservation}] A python package for managing and
  manipulating mock observations for the Event Horizon Telescope
  (\url{http://github.com/chanchikwan/mockservation})

\item[\texttt{RosettaTour}] A virtual reality mobile app compatible
  with Google Cardboard for touring the Rosetta mission
  (\url{http://github.com/AstroCardboard/RosettaTour})

\item[\texttt{insight}] An OpenGL-based interactive virtual reality
  data visualizer for Oculus Rift
  (\url{http://github.com/chanchikwan/insight})

\item[\texttt{gray}] A massive parallel ODE integrator for performing
  general relativistic radiative transfer using ray tracing
  (\url{http://github.com/chanchikwan/gray})

\item[\texttt{orbits}] A collection of symplectic integrators that are
  ideal for solving celestial mechanic problems
  (\url{http://github.com/chanchikwan/orbits})

\item[\texttt{fg2}] A 2D grid-based partial differential equation
  solver written in CUDA~C and runs on nVidia GPUs
  (\url{http://github.com/chanchikwan/fg2})

\item[\texttt{sg2}] A 2D spectral Galerkin code written in CUDA~C and
  runs on nVidia GPUs (\url{http://github.com/chanchikwan/sg2})

\end{Tlist}

\noindent See my GitHub page (\url{http://github.com/chanchikwan}) for
additional open source software projects
